
% JuliaCon proceedings template
\documentclass{juliacon}
\setcounter{page}{1}

\begin{document}

% **************GENERATED FILE, DO NOT EDIT**************

\title{TSML (Time Series Machine Learning)}

\author{Paulito P. Palmes}
\author{Joern Ploennigs}
\author{Niall Brady}
\affil{IBM Dublin Research Lab}

\keywords{Julia, Time Series, Machine Learning, Filters, Feature Extraction, Classification, Prediction, Aggregation, Imputation}



\maketitle

\begin{abstract}

  Over the past years, the industrial sector has seen many innovations brought about by automation. 
  Inherent in this automation is the installation of sensor networks for status monitoring and data collection. 
  One of the major challenges in these data-rich environments is how to extract and exploit information from 
  these large volume of data to detect anomalies, discover patterns to reduce downtimes and manufacturing 
  errors, reduce energy usage, predict faults/failures, effective maintenance schedules, etc. 
  To address these issues, we developed TSML. Its technology is based 
  on using the pipeline of lightweight filters as building blocks to process huge amount of industrial time series data in parallel.  

\end{abstract}

\section{Introduction}

\textbf{TSML} is a package for time series data processing, classification, and prediction. 
It provides common API for ML libraries from Python's ScikitLearn, 
R's caret, and native Julia MLs for seamless integration of heterogeneous 
libraries to create complex ensembles for robust time-series prediction, clustering, and classification.
TSML has the following features:
\begin{itemize}
\item data type clustering/classification for automatic data discovery
\item aggregation based on date/time interval
\item imputation based on symmetric Nearest Neighbors
\item statistical metrics for data quality assessment
\item ML wrapper with more than 100+ libraries from caret, scikitlearn, and julia
\item date/value matrix conversion of 1-D time series using sliding windows for ML input
\item pipeline API for high-level description of the processing workflow
\item specific cleaning/normalization workflow based on data type
\item automatic selection of optimised ML model
\item automatic segmentation of time-series data into matrix form for ML training and prediction 
\item extensible architecture using just two main interfaces: fit and transform
\item meta-ensembles for automatic feature and model selection
\item support for distributed computation for scalability and speed
\end{itemize}

The \textbf{TSML} package assumes a two-column input composed of \emph{dates} and \emph{values}. 
The first part of the workflow aggregates values based on the specified date/time 
interval which minimizes occurrence of missing values and noise. The aggregated 
data is then left-joined to the complete sequence of dates in a specified date/time interval. 
Remaining missing values are replaced by \textit{k} nearest neighbors where \textit{k} is the symmetric 
distance from the location of missing value. This approach can be called several 
times until there are no more missing values.

The next part of the workflow extracts the date features and 
convert the value column into matrix form parameterized by 
the size and stride of the sliding window. The final part joins
 the date features and the value matrix to serve as input to the 
 ML with the output representing the values of the time periods 
 to be predicted ahead of time.
 
\textbf{TSML} uses a pipeline which iteratively calls the \emph{fit} and \emph{transform}
families of functions relying on `multiple dispatch` to select the correct 
algorithm from the steps outlined above. Machine learning functions in 
\textbf{TSML} are wrappers to the corresponding Scikit-learn, Caret, and native Julia ML libraries. 
There are more than hundred classifiers and regression functions available using a common API.

\section{TSML Workflow}
\label{sec:tsmlworkflow}
%

\textbf{TSML} workflow borrows the idea of Unix pipeline.
The main elements in a pipeline are series of filters
each performing one specific task and does it well. 

\vskip 6pt
To illustrate, below describes the main steps in using \textbf{TSML}.
First, we create filters for csv reading, aggregation, imputation, data quality
assessment.

\begin{lstlisting}[language = Julia]
fname = joinpath(dirname(pathof(TSML)),
   "../data/testdata.csv")
csvfilter = DataReader(Dict(
   :filename=>fname,
   :dateformat=>"dd/mm/yyyy HH:MM"))
valgator = DateValgator(Dict(
   :dateinterval=>Dates.Hour(1)))
valnner = DateValNNer(Dict(
   :dateinterval=>Dates.Hour(1)))
stfier = Statifier(Dict(:processmissing=>true))
\end{lstlisting}

We can then setup a pipeline containing these filters to process the csv data
by aggregating the time series hourly and check the data quality using the
\emph{Statifier} filter.

\begin{lstlisting}[language = Julia]
apipeline = Pipeline(Dict(
   :transformers => [csvfilter, valgator, stfier]))
fit!(apipeline)
mystats = transform!(apipeline)
@show mystats
\end{lstlisting}

Calling the \emph{fit} and \emph{transform} in the pipeline
iteratively calls the corresponding \emph{fit} and \emph{transform} within each filter. 
This common API relying on Julia's multi-dispatch mechanism greatly simplifies the implementations, operations, 
and understanding of the entire workflow. In addition, extending TSML functionality is just a 
matter of creating a new data type filter and define its own  \emph{fit} and \emph{transform} 
functions.

In the \emph{Statifier} filter result, blocks of missing data is indicated by column names starting
with \emph{b}. Running the code above indicates that there are plenty of missing data blocks.
We can add the \emph{ValNNer} filter to perform \emph{k} nearest neighbour imputation and check
the statistics:

\begin{lstlisting}[language = Julia]
bpipeline = Pipeline(Dict(
  :transformers => [csvfilter, valgator, 
                    valnner,stfier]))
fit!(bpipeline)
imputed = transform!(bpipeline)
@show imputed
\end{lstlisting}

The result now indicates \emph{NaN} for all missing data statistics column because the set 
of missing blocks count is now empty.

\section{Time Series Classification}

We can now use the knowledge we learned in setting up 
\emph{pipeline} and \emph{filters} to build higher level
operations to solve a specific industrial problem. One major problem
which we consider relevant because it is a common issue in IOT (Internet of Things) 
 is the time series classification. This problem is prevalent nowadays 
due to the increasing need to use many sensors to monitor status in different aspects of industrial
operations and maintenance of cars, buildings, hospitals, supermarkets, homes, cities, etc.

Rapid deployment of these sensors result to many of them not properly labeled or classified.
Time series classification is a significant first step for optimal prediction and anomaly detection.
To successfully perform the latter operations, it is necessary to identify first the time series
type so that appropriate model and cleaning routines can be selected for optimal model performance . 
The  \emph{TSClassifier} filter aims to address this problem and its usage is described below.

\vskip 6pt
First, we setup the locations of files for training, testing, and saving the model.
Next, we start the training phase by calling \emph{fit} which loads
file in the training directory and learn the mapping between their
statistic features extracted by \emph{Statifier} with their types indicated
by their filenames. Once the training is done, the final model
is saved in the \emph{model} directory which will be used for 
testing accuracy and classifying new time series datasets. 

\begin{lstlisting}[language = Julia]
trdirname = "training"
tstdirname = "testing"
modeldirname = "model"
tscl = TSClassifier(Dict(
   :trdirectory=>trdirname,
   :tstdirectory=>tstdirname,
   :modeldirectory=>modeldirname,
   :num_trees=>50)
)
fit!(tscl)
predictions = transform!(tscl)
@show testingAccuracy(predictions)
\end{lstlisting}

\section{Extending TSML with Scikitlearn and Caret}
In the latest \textbf{TSML} version (2.3.4), we refactored the base TSML
to only include pure Julia code implementations and moved
external libs and binary dependencies into the TSMLextra package. 
One major reason is to have a smaller code base so that it can be easily
maintained and rapidly deployed. Moreover, smaller codes make
static compilation fast for smaller docker image  
in \emph{Kubernetes} deployment. 

There are cases however where the main task of time series classification 
requires more complex ensemble model using hierarchy or tree where 
members are composed of heterogeneous ML learners from binaries in 
different languages. For illustration purposes, we will show how to 
ensemble ML libraries from ScikitLearn and Caret using \textbf{TSML} 
meta-ensembles that support the \emph{fit} and \emph{transform} APIs.

\vskip 6pt

Load necessary modules:
\begin{lstlisting}[language = Julia]
@everywhere using TSML, TSMLextra, Plots
@everywhere using TSML.TSMLTypes
@everywhere using TSML: TSClassifier
@everywhere using TSML.TSClassifiers.FileStats
@everywhere using TSML.TSMLTransformers
@everywhere using TSML.EnsembleMethods
@everywhere using TSML.DecisionTreeLearners
@everywhere using TSML.Utils
@everywhere using TSMLextra.CaretLearners
@everywhere using TSMLextra.SKLearners
@everywhere using TSML.MLBaseWrapper

@everywhere using DataFrames
@everywhere using Random
@everywhere using Statistics
@everywhere using StatsBase: iqr
@everywhere using RDatasets
\end{lstlisting}

Setup external learners:
\begin{lstlisting}[language = Julia]
# Caret ML
@everywhere caret_svmlinear = 
   CaretLearner(Dict(:learner=>"svmLinear"))
@everywhere caret_treebag = 
   CaretLearner(Dict(:learner=>"treebag"))
@everywhere caret_rpart = 
   CaretLearner(Dict(:learner=>"rpart"))
@everywhere caret_rf = 
   CaretLearner(Dict(:learner=>"rf"))

# ScikitLearn ML
@everywhere sk_ridge = 
   SKLearner(Dict(:learner=>"RidgeClassifier"))
@everywhere sk_sgd = 
   SKLearner(Dict(:learner=>"SGDClassifier"))
@everywhere sk_knn = 
   SKLearner(Dict(:learner=>"KNeighborsClassifier"))
@everywhere sk_gb = 
   SKLearner(Dict(:learner=>
   "GradientBoostingClassifier",
   :impl_args=>Dict(:n_estimators=>10)))
@everywhere sk_extratree = 
   SKLearner(Dict(:learner=>"ExtraTreesClassifier",
   :impl_args=>Dict(:n_estimators=>10)))
@everywhere sk_rf = 
   SKLearner(Dict(:learner=>
   "RandomForestClassifier",
   :impl_args=>Dict(:n_estimators=>10)))
\end{lstlisting}

Setup Julia learners and meta-ensembles:
\begin{lstlisting}[language = Julia]
# Julia ML
@everywhere jrf = RandomForest()
@everywhere jpt = PrunedTree()
@everywhere jada = Adaboost()

# Julia Ensembles
@everywhere jvote_ens=VoteEnsemble(Dict(
   :learners=>[jrf,jpt,sk_gb,sk_extratree,sk_rf]))
@everywhere jstack_ens=StackEnsemble(Dict(
   :learners=>[jrf,jpt,sk_gb,sk_extratree,sk_rf]))
@everywhere jbest_ens=BestLearner(Dict(
   :learners=>[jrf,sk_gb,sk_rf]))
@everywhere jsuper_ens=VoteEnsemble(Dict(
   :learners=>[jvote_ens,jstack_ens,
               jbest_ens,sk_rf,sk_gb]))
\end{lstlisting}

Setup pipeline for training and prediction:
\begin{lstlisting}[language = Julia]
@everywhere function predict(learner,
            data,train_ind,test_ind)
            
  features = convert(Matrix,data[:, 1:(end-1)])
  labels = convert(Array,data[:, end])
  # Create pipeline
  pipeline = Pipeline(
    Dict(
      :transformers => [
        OneHotEncoder(), # nominal to bits
        Imputer(), # Imputes NA values
        StandardScaler(), # normalize
        learner # Predicts labels on instances
      ]
    )
  )
  
  # Train
  fit!(pipeline, features[train_ind, :],
       labels[train_ind])
       
  # Predict
  predictions = transform!(pipeline, 
      features[test_ind, :])
  
  # Assess predictions
  result = score(:accuracy, 
      labels[test_ind], predictions)
  return result
end
\end{lstlisting}


TODO: 
\begin{itemize}
\item add plots
\item add stats output
\item add more examples from docs
\end{itemize}
%\begin{verbatim}
%\bibliographystyle{juliacon}
%\bibliography{ref}
%\end{verbatim}
%When submitting the document source (.tex) file to external
%parties, the ref.bib file should be sent with it.
%\cite{bezanson2017julia}

%% **************GENERATED FILE, DO NOT EDIT**************

\bibliographystyle{juliacon}
\bibliography{ref.bib}


\end{document}

% Inspired by the International Journal of Computer Applications template
